\PassOptionsToPackage{unicode=true}{hyperref} % options for packages loaded elsewhere
\PassOptionsToPackage{hyphens}{url}
%
\documentclass[
]{article}
\usepackage{lmodern}
\usepackage{amssymb,amsmath}
\usepackage{ifxetex,ifluatex}
\ifnum 0\ifxetex 1\fi\ifluatex 1\fi=0 % if pdftex
  \usepackage[T1]{fontenc}
  \usepackage[utf8]{inputenc}
  \usepackage{textcomp} % provides euro and other symbols
\else % if luatex or xelatex
  \usepackage{unicode-math}
  \defaultfontfeatures{Scale=MatchLowercase}
  \defaultfontfeatures[\rmfamily]{Ligatures=TeX,Scale=1}
\fi
% use upquote if available, for straight quotes in verbatim environments
\IfFileExists{upquote.sty}{\usepackage{upquote}}{}
\IfFileExists{microtype.sty}{% use microtype if available
  \usepackage[]{microtype}
  \UseMicrotypeSet[protrusion]{basicmath} % disable protrusion for tt fonts
}{}
\makeatletter
\@ifundefined{KOMAClassName}{% if non-KOMA class
  \IfFileExists{parskip.sty}{%
    \usepackage{parskip}
  }{% else
    \setlength{\parindent}{0pt}
    \setlength{\parskip}{6pt plus 2pt minus 1pt}}
}{% if KOMA class
  \KOMAoptions{parskip=half}}
\makeatother
\usepackage{xcolor}
\IfFileExists{xurl.sty}{\usepackage{xurl}}{} % add URL line breaks if available
\IfFileExists{bookmark.sty}{\usepackage{bookmark}}{\usepackage{hyperref}}
\hypersetup{
  pdftitle={Statistical appendix: Derivation of Confidence Intervals},
  pdfauthor={R-Walmsley},
  pdfborder={0 0 0},
  breaklinks=true}
\urlstyle{same}  % don't use monospace font for urls
\usepackage[margin=1in]{geometry}
\usepackage{graphicx,grffile}
\makeatletter
\def\maxwidth{\ifdim\Gin@nat@width>\linewidth\linewidth\else\Gin@nat@width\fi}
\def\maxheight{\ifdim\Gin@nat@height>\textheight\textheight\else\Gin@nat@height\fi}
\makeatother
% Scale images if necessary, so that they will not overflow the page
% margins by default, and it is still possible to overwrite the defaults
% using explicit options in \includegraphics[width, height, ...]{}
\setkeys{Gin}{width=\maxwidth,height=\maxheight,keepaspectratio}
\setlength{\emergencystretch}{3em}  % prevent overfull lines
\providecommand{\tightlist}{%
  \setlength{\itemsep}{0pt}\setlength{\parskip}{0pt}}
\setcounter{secnumdepth}{-2}
% Redefines (sub)paragraphs to behave more like sections
\ifx\paragraph\undefined\else
  \let\oldparagraph\paragraph
  \renewcommand{\paragraph}[1]{\oldparagraph{#1}\mbox{}}
\fi
\ifx\subparagraph\undefined\else
  \let\oldsubparagraph\subparagraph
  \renewcommand{\subparagraph}[1]{\oldsubparagraph{#1}\mbox{}}
\fi

% set default figure placement to htbp
\makeatletter
\def\fps@figure{htbp}
\makeatother


\title{Statistical appendix: Derivation of Confidence Intervals}
\author{R-Walmsley}
\date{03/01/2020}

\begin{document}
\maketitle

\hypertarget{derivation-of-confidence-intervals}{%
\section{Derivation of Confidence
Intervals}\label{derivation-of-confidence-intervals}}

Note that throughout when referring to confidence intervals, the
two-sided 95\% confidence interval is used. The theory easily
generalises, but only the 95\% confidence interval is currently
implemented.

\hypertarget{plotting-results-of-compositional-data-analyses}{%
\subsection{Plotting results of Compositional Data
Analyses}\label{plotting-results-of-compositional-data-analyses}}

Here, the results of Compositional Data Analyses are presented
graphically, using plots of model predictions at given pairwise balances
of parts (\texttt{plot\_transfers}) as well as model predictions at
particular compositions (\texttt{forest\_plot\_comp}). Where results are
presented, there are two options for how to do this. The default is to
set \texttt{terms\ =\ TRUE}, and present the predicted difference in the
outcome at the different levels of the compositional variables.
Implicitly, in this case, all other covariates are fixed.

The alternative is to set \texttt{terms\ =\ FALSE}. In this case, the
\texttt{fixed\_values} argument is used (if \texttt{fixed\_values} is
\texttt{NULL}, the default, the \texttt{generate\_fixed\_values}
function is used to set this argument internally, using median, modal
and compositional mean values as appropriate). The plot shows model
predictions for the specified composition and for additional
(non-compositional) covariate values as given in \texttt{fixed\_values}.

\hypertarget{confidence-intervals-when-terms-true}{%
\subsection{\texorpdfstring{Confidence intervals when
\texttt{terms\ =\ TRUE}}{Confidence intervals when terms = TRUE}}\label{confidence-intervals-when-terms-true}}

In all cases, using \texttt{predict} with \texttt{type\ =\ "terms"}
gives fitted values of each specified term on the linear predictor scale
i.e.~for \(x_1\) it returns \(\beta_1 x_1\). If the (transformed)
compositional variables are \(z_1, z_2, ... z_n\) and these are the
specified terms, then \texttt{predict} returns estimates of
\(\beta_1 z_1, \beta_2 z_2, ... \beta_n z_n\) which can be used to
calculate \$ \widehat{\Delta y} \beta\_1 z\_1 + \beta\_2 z\_2 + \ldots{}
+ \beta\_n z\_n \$. We now need to calculate the uncertainty on this
quantity.

\hypertarget{linear-models}{%
\subsubsection{Linear models}\label{linear-models}}

This is the simplest case. Using the above, we directly estimate \$
\widehat{\Delta y} \beta\_1 z\_1 + \beta\_2 z\_2 + \ldots{} + \beta\_n
z\_n \$ and the confidence interval on this,
\((\widehat{\Delta y_{min}}, \widehat{\Delta y_{max}})\).

\hypertarget{logistic-models}{%
\subsubsection{Logistic models}\label{logistic-models}}

Here, the terms given by the \texttt{predict} function are on the scale
of the linear predictors i.e.~we estimate \$ \widehat{\Delta y} =
\beta\_1 z\_1 + \beta\_2 z\_2 + \ldots{} + \beta\_n z\_n \$ and the
confidence interval on this,
\((\widehat{\Delta y_{min}}, \widehat{\Delta y_{max}})\). Then, the odds
ratio and uncertainty on it is given by exponentiating the estimate
(\(\exp(\widehat{\Delta y})\)) and confidence interval
(\((\exp(\widehat{\Delta y_{min}}), \exp(\widehat{\Delta y_{max}}))\).

\hypertarget{cox-models}{%
\subsubsection{Cox models}\label{cox-models}}

Here, the terms given by the \texttt{predict} function are on the scale
of the linear predictors i.e.~we estimate \$ \widehat{\Delta y} =
\beta\_1 z\_1 + \beta\_2 z\_2 + \ldots{} + \beta\_n z\_n \$ and the
confidence interval on this,
\((\widehat{\Delta y_{min}}, \widehat{\Delta y_{max}})\). Then, the
hazard ratio and uncertainty on it is given by exponentiating the
estimate (\(\exp(\widehat{\Delta y})\)) and confidence interval
(\((\exp(\widehat{\Delta y_{min}}), \exp(\widehat{\Delta y_{max}}))\).

\hypertarget{comparison-with-existing-literature}{%
\subsubsection{Comparison with existing
literature}\label{comparison-with-existing-literature}}

Note that the treatment here takes inspiration from but slightly
deviates from Dumuid et al (Dumuid et al. 2017). Dumuid et al deals with
the case where one compositional part is considered at the expense of
all others proportionately. By construction of the ilr pivot
coordinates, the uncertainty on the estimate in this case can be derived
entirely from the first pivot coordinate. More generally, however, and
particularly when wishing to consider arbitrary compositions (as in the
case of \texttt{forest\_plot\_comp}), this is not the case. Therefore,
here we use the uncertainty on all compositional variables when deriving
the estimated confidence intervals. This will never give a narrower
confidence interval than approaches using uncertainty on only some
compositional terms; it may give a wider (more conservative) confidence
interval in some cases, as it considers the uncertainty on compositional
variables that are in fact fixed when considering just a particular
comparison of compositions/ pairwise balance. This is fully general, and
fits well conceptually with considering the exposure of interest as the
overall composition (perturbed and perturbable in various ways).

-- a note on using balances. ** need to think more about this. -- need
to compare systematically different approaches and do the derivation. To
consider pairwise behaviour balances, \#\#\# Confidence intervals when
\texttt{terms\ =\ FALSE} In this case, for linear, logistic and Cox
models, the confidence intervals are the standard confidence intervals
on the mean prediction,as calculated by the \texttt{predict} method.
Note the intervals are \emph{not} the prediction intervals (the
intervals within)

\[
\sqrt{27}
\]

\hypertarget{refs}{}
\leavevmode\hypertarget{ref-Dumuid2017c}{}%
Dumuid, Dorothea, Željko Pedišić, Tyman Everleigh Stanford, Josep Antoni
Martín-Fernández, Karel Hron, Carol A. Maher, Lucy K. Lewis, and Timothy
Olds. 2017. ``The compositional isotemporal substitution model: A method
for estimating changes in a health outcome for reallocation of time
between sleep, physical activity and sedentary behaviour.''
\url{https://doi.org/10.1177/0962280217737805}.

\end{document}
